\documentclass[12pt,a4paper]{article}
\usepackage[utf8]{inputenc}
\usepackage[T1]{fontenc}
\usepackage{amsmath,amssymb,amsthm}
\usepackage{geometry}
\usepackage{hyperref}
\usepackage{booktabs}

\geometry{margin=1in}

\newtheorem{theorem}{Theorem}
\newtheorem{axiom}{Axiom}
\newtheorem{proposition}{Proposition}

\title{\textbf{Gravity as Constraint:}\\[6pt]
Self-Consistency as Physical Law,\\
With Fixed-Point Existence Proofs and Decoherence Bounds}

\author{Will [Last Name]\thanks{Corresponding author.} \quad Claude (Anthropic)\thanks{AI assistant; contributed to mathematical development and analysis.}}

\date{February 2026}

\begin{document}
\maketitle

\begin{abstract}
We propose that physical law is what self-consistency looks like: the universe is the fixed point of a constraint requiring that geometry and the quantum fields it hosts mutually determine each other. Gravity is not an independent degree of freedom to be quantized but a constraint---the demand that the block spacetime be self-consistent. The semiclassical Einstein equation, together with its stochastic extension via the noise kernel, implements this constraint at tree level in the stochastic gravity formalism of Hu and Verdaguer.

We prove that self-consistent solutions exist: exactly in cosmology via the trace anomaly, perturbatively near any classical solution via Banach contraction with constant $\kappa \sim (m/M_P)^2 \ll 1$, and conditionally in the general case via the Schauder theorem. The Planck scale emerges as the natural validity boundary where $\kappa \to 1$.

The stochastic extension yields Gaussian gravitational decoherence at a timescale $\tau_{\text{coh}} \approx 1.13\,\hbar/E_\Delta$, where $E_\Delta$ is the gravitational self-energy of the superposed mass distribution---the same energy that determines the Di\'osi--Penrose timescale. The temporal profile (Gaussian rather than exponential) is the distinguishing prediction: it arises because the noise kernel for stationary matter is time-independent in the Newtonian limit, unlike the white noise postulated in the Di\'osi model. The framework further predicts null results for BMV entanglement at the quantum gravity rate. Both predictions are falsifiable.
\end{abstract}

%======================================================================
\section{Motivation}
%======================================================================

Consider the possibility that physical law is self-consistency. Not a dynamical rule imposed from outside, but the demand that the universe---the complete block spacetime and the quantum fields it hosts---be a fixed point: every part determining every other part with no remainder and no contradiction.

If this is taken seriously, gravity is not a force. It is a constraint. The geometry of spacetime is whatever it must be for the quantum fields propagating on it to source, through their stress-energy, exactly that geometry. The block is the ontology: the complete four-dimensional solution, not a three-dimensional state evolving forward. The arrow of time is a geometric feature of the block---the direction in which entropy increases---not a primitive dynamical ingredient.

This view has concrete mathematical content. The semiclassical Einstein equation is precisely a self-consistency condition: the geometry and the quantum state must be mutually compatible. Its stochastic extension via the noise kernel promotes this from a mean-field constraint to one that includes quantum fluctuations. Together, they constitute tree-level quantum gravity expressed in the stochastic gravity formalism of Hu and Verdaguer~\cite{hu_verdaguer}, which reproduces the full spectrum of inflationary perturbations without quantizing the metric~\cite{roura_verdaguer}.

The position that the semiclassical Einstein equation may be fundamental dates to M\o ller (1962) and Rosenfeld (1963), and has been developed in its most sophisticated modern form by Oppenheim~\cite{oppenheim}. What this paper adds to the program is:
\begin{enumerate}
\item Rigorous fixed-point existence proofs establishing that the self-consistency equation has solutions, with a quantitative contraction constant $\kappa \sim (m/M_P)^2$ and a natural Planck-scale validity boundary.
\item A derivation of the gravitational decoherence rate that actually follows from semiclassical self-consistency, which turns out to be much slower than the commonly cited Di\'osi--Penrose rate.
\item A demonstration that the Di\'osi--Penrose timescale requires physics beyond the semiclassical Einstein equation---a result that sharpens the theoretical landscape for gravitational decoherence experiments.
\end{enumerate}

%======================================================================
\section{Core Proposal}
%======================================================================

\subsection{The Self-Consistency Equation}

The fundamental equation of the framework is:
\begin{equation}
G_{\mu\nu}[g] = 8\pi G \, \langle \Psi | \hat{T}_{\mu\nu}[g] | \Psi \rangle
\label{eq:sce}
\end{equation}
where $G_{\mu\nu}[g]$ is the Einstein tensor computed from the geometry $g_{\mu\nu}$, $|\Psi\rangle$ is the global quantum state of all matter fields, and $\hat{T}_{\mu\nu}[g]$ is the stress-energy tensor operator defined on the geometry $g$. The expectation value is computed on the same geometry that the left side describes.

This is not a dynamical equation in the usual sense. It is a fixed-point condition: we seek the geometry $g_{\mu\nu}$ such that the quantum state it hosts, evolving on it, sources exactly that geometry.

\subsection{Axioms}

\begin{axiom}[Ontology]
The fundamental entities are quantum fields $\hat{\phi}_i$ defined on a 4-dimensional Lorentzian manifold $(\mathcal{M}, g_{\mu\nu})$.
\end{axiom}

\begin{axiom}[Geometry as constraint]
The geometry is not independently dynamical. Its mean is determined by the self-consistency condition~\eqref{eq:sce}; its fluctuations are determined by the noise kernel of quantum stress-energy (eq.~\ref{eq:noise_kernel}). Both the mean geometry and its stochastic corrections are sourced entirely by quantum matter---no independent gravitational degrees of freedom contribute. The noise kernel carries quantum correlations of the stress-energy; at tree level, the resulting stochastic metric fluctuations are isomorphic to graviton exchange~\cite{hu_verdaguer}. The framework therefore constitutes tree-level quantum gravity expressed in stochastic language.
\end{axiom}

\begin{axiom}[Global solution]
The triple $(\mathcal{M}, g_{\mu\nu}, |\Psi\rangle)$ is solved as a fixed-point problem over the entire manifold. The solution is the block---the complete spacetime.
\end{axiom}

%======================================================================
\section{Fixed-Point Existence}
\label{sec:fixedpoint}
%======================================================================

This section contains the framework's core mathematical contribution. The self-consistency equation~\eqref{eq:sce} is a fixed-point condition; the question of whether the framework is well-defined reduces to whether fixed points exist. We establish existence at three levels of generality.

\subsection{The Map}

Define $\mathcal{F}: g_{\mu\nu} \mapsto g'_{\mu\nu}$ by:
\begin{enumerate}
\item Given a trial geometry $g$, construct the QFT on $(\mathcal{M}, g)$ and identify the physical state $|\Psi_g\rangle$ via the Hadamard condition.
\item Compute $S_{\mu\nu}[g] \equiv 8\pi G \langle \hat{T}_{\mu\nu} \rangle_{g,\text{ren}}$.
\item Solve $G_{\mu\nu}[g'] = S_{\mu\nu}[g]$ for $g'$.
\end{enumerate}
We seek $g^*$ such that $\mathcal{F}(g^*) = g^*$.

\subsection{Cosmological Fixed Point: Exact Solution}

\begin{theorem}[Starobinsky, 1980 \cite{starobinsky}]
The self-consistency equation on FRW spacetime with conformal matter admits an exact de~Sitter solution $a(t) = a_0 e^{H_0 t}$ with:
\begin{equation}
H_0^2 = \frac{180\pi}{G |a_2|}
\end{equation}
where $a_2$ depends on the field content. This solution is a stable attractor: perturbations $\delta H$ are exponentially damped.
\end{theorem}

\noindent\emph{Proof sketch.} For constant $H$, all time derivatives vanish. The trace anomaly~\cite{capper_duff} reduces to $\langle \hat{T}^\mu{}_\mu \rangle = -a_2 H_0^4 / 120\pi^2$. The Friedmann equation yields $H_0$. Stability follows from the linearized equation for $\delta H$~\cite{starobinsky}.

This constitutes an existence proof: the de~Sitter geometry, when used to compute the quantum stress-energy of fields living on it, sources exactly itself.

\subsection{Perturbative Fixed Point: Banach Contraction}

Let $\bar{g}$ be a solution of the classical Einstein equations. Define the quantum-corrected map:
\begin{equation}
\mathcal{F}(g) = \bar{g} + (G^{\bar{g}}_{\text{lin}})^{-1} \left[ 8\pi G \langle \hat{T}_{\mu\nu} \rangle^{(\text{quantum})}_g \right]
\end{equation}

\begin{theorem}[Perturbative existence]
The map $\mathcal{F}$ is a contraction in the $H^s$ norm with contraction constant:
\begin{equation}
\kappa \sim \frac{1}{2\pi} \left( \frac{m}{M_P} \right)^2 \ell_P^2 R
\label{eq:kappa}
\end{equation}
where $m$ is the field mass, $M_P = 1/\sqrt{G}$ is the Planck mass, and $R$ is the background curvature scale.
\end{theorem}

\noindent\emph{Proof sketch.} The Lipschitz constant of $\mathcal{F}$ is bounded by $\kappa = 8\pi G \| (G_{\text{lin}})^{-1} \|_{\text{op}} \cdot \| \delta \langle \hat{T}_{\mu\nu} \rangle / \delta g \|_{\text{op}}$. The response kernel has been computed~\cite{horowitz,phillips_hu}; its operator norm scales as $\hbar m^2 c / (4\pi)^2$. Combined with the inverse linearized Einstein operator norm $\sim 1/R$, this yields the result.

For all known particles, $\kappa \ll 1$:

\medskip
\begin{center}
\begin{tabular}{lcc}
\toprule
Particle & Mass (kg) & $\kappa \sim (m/M_P)^2$ \\
\midrule
Electron & $9.1 \times 10^{-31}$ & $\sim 10^{-45}$ \\
Proton & $1.7 \times 10^{-27}$ & $\sim 10^{-39}$ \\
Top quark & $3.1 \times 10^{-25}$ & $\sim 10^{-34}$ \\
\bottomrule
\end{tabular}
\end{center}
\medskip

By the Banach fixed-point theorem, a unique self-consistent solution exists in a neighborhood of any classical solution, and the iterative scheme converges exponentially.

\subsection{Non-Perturbative Existence via Schauder}

For strong-field regimes, we invoke the Schauder fixed-point theorem~\cite{schauder}. If $\mathcal{F}$ maps a convex, compact subset $K$ of the metric space into itself, and $\mathcal{F}$ is continuous, then $\mathcal{F}$ has a fixed point. Continuity follows from the continuous dependence of the Hadamard parametrix on the metric~\cite{radzikowski} and the well-posedness of the Einstein equations~\cite{choquet_bruhat}. Compactness of $K$ in the appropriate Sobolev space follows from the Rellich--Kondrachov theorem.

We note that a fully rigorous Schauder argument requires specifying the convex compact subset (including gauge fixing) and verifying $\mathcal{F}$-invariance---nontrivial steps in Lorentzian signature with diffeomorphism invariance. We regard this as a conditional result pending further functional-analytic development.

\subsection{Breakdown at the Planck Scale}

The analysis fails when $R \sim \ell_P$: the contraction constant $\kappa$ approaches 1, and neither the perturbative nor the Schauder argument applies. The Planck scale emerges as the natural boundary of the framework's validity---not imposed by hand, but derived from the mathematics of fixed-point convergence.

%======================================================================
\section{Gravitational Decoherence from Self-Consistent Fluctuations}
\label{sec:decoherence}
%======================================================================

We now derive the gravitational decoherence rate that follows from the self-consistency framework. This requires careful analysis, as the result differs from what has been widely assumed.

\subsection{The Deterministic Equation Does Not Decohere}

\begin{proposition}
The deterministic self-consistency equation~\eqref{eq:sce} does not produce decoherence for spatially separated superpositions.
\end{proposition}

\noindent\emph{Proof.} Consider a mass $m$ in superposition $|\psi\rangle = \frac{1}{\sqrt{2}}(|L\rangle + |R\rangle)$, with $|L\rangle$ and $|R\rangle$ spatially separated by $d$. The density matrix is:
\begin{equation}
\rho = \frac{1}{2}\left(|L\rangle\langle L| + |R\rangle\langle R| + |L\rangle\langle R| + |R\rangle\langle L|\right)
\end{equation}
The self-consistency equation sources the geometry via $\langle \hat{T}_{\mu\nu} \rangle = \mathrm{Tr}(\rho \, \hat{T}_{\mu\nu})$. For spatially separated wavepackets, the stress-energy operator is local: $\langle L | \hat{T}_{\mu\nu}(x) | R \rangle \approx 0$ when $x$ is in the support of either wavepacket. Therefore:
\begin{equation}
\langle \hat{T}_{\mu\nu} \rangle \approx \frac{1}{2} T^{(L)}_{\mu\nu} + \frac{1}{2} T^{(R)}_{\mu\nu}
\label{eq:tmunu_avg}
\end{equation}
independent of the off-diagonal coherence $\rho_{LR}$. The geometry determined by eq.~\eqref{eq:sce} is therefore identical whether the system is in a coherent superposition or a classical mixture. Since the Hamiltonian governing the matter evolution depends only on this geometry, the time evolution of $\rho_{LR}$ is purely oscillatory (a phase rotation), not decaying. The deterministic self-consistency equation is blind to coherence and cannot destroy it. \hfill$\square$

This result has a simple physical interpretation: the self-consistency equation extracts $\langle \hat{T}_{\mu\nu} \rangle$ from the quantum state and feeds it back as a classical geometry. For separated wavepackets, this expectation value carries no information about the quantum coherence between branches. A classical geometry that carries no branch-distinguishing information cannot selectively suppress the off-diagonal elements of $\rho$.

\subsection{The Stochastic Extension}

The semiclassical equation uses the mean $\langle \hat{T}_{\mu\nu} \rangle$, but quantum operators fluctuate. The natural next-order correction incorporates these fluctuations through the Einstein--Langevin equation~\cite{hu_verdaguer}:
\begin{equation}
G_{\mu\nu}[g] = 8\pi G \left( \langle \hat{T}_{\mu\nu} \rangle_g + \xi_{\mu\nu} \right)
\label{eq:einstein_langevin}
\end{equation}
where $\xi_{\mu\nu}$ is a stochastic tensor with zero mean and correlator given by the noise kernel:
\begin{equation}
\langle \xi_{\mu\nu}(x) \, \xi_{\alpha\beta}(x') \rangle = N_{\mu\nu\alpha\beta}(x, x') \equiv \frac{1}{2} \langle \{ \delta\hat{T}_{\mu\nu}(x), \, \delta\hat{T}_{\alpha\beta}(x') \} \rangle
\label{eq:noise_kernel}
\end{equation}
with $\delta\hat{T}_{\mu\nu} \equiv \hat{T}_{\mu\nu} - \langle \hat{T}_{\mu\nu} \rangle$.

This is not an additional postulate. It is the leading fluctuation correction to any mean-field theory, analogous to the passage from Hartree to Hartree-Fock in atomic physics or from the Boltzmann equation to the Langevin equation in statistical mechanics. The noise kernel $N_{\mu\nu\alpha\beta}$ is computable from the quantum state and the geometry~\cite{phillips_hu}.

\subsection{Decoherence from Stochastic Geometry}
\label{sec:decoherence_derivation}

The stochastic metric fluctuations induce a random potential that differs between branches. For a mass $m$ in superposition of size $d$, the Newtonian potential fluctuation at each branch location is $\delta\Phi_{L,R} = -(c^2/2) \, \delta g_{00}(x_{L,R})$. The off-diagonal element of the density matrix, averaged over stochastic realizations, evolves as:
\begin{equation}
|\rho_{LR}(t)| = |\rho_{LR}(0)| \, \exp\!\left( -\frac{1}{2\hbar^2} \int_0^t \!\!\int_0^t C_V(t' - t'') \, dt' \, dt'' \right)
\label{eq:decoherence_gaussian}
\end{equation}
where $C_V(\tau) \equiv \langle [\delta V_L(t) - \delta V_R(t)][\delta V_L(t+\tau) - \delta V_R(t+\tau)] \rangle$ is the potential difference correlator, $\delta V_{L,R} = m \, \delta\Phi_{L,R}$ are the stochastic potential fluctuations at the two branch locations, and the Gaussian average over noise realizations gives the exponential form.

The stochastic metric perturbation satisfies the linearized Einstein equation $G^{\text{lin}}_{\mu\nu}[\delta g] = 8\pi G \, \xi_{\mu\nu}$, where $\xi_{\mu\nu}$ is the stochastic source with correlator~\eqref{eq:noise_kernel}. In the Newtonian limit, this reduces to the stochastic Poisson equation:
\begin{equation}
\nabla^2 \delta\Phi(\mathbf{x}, t) = \frac{4\pi G}{c^2} \, \xi_{00}(\mathbf{x}, t)
\end{equation}
whose Green function $G(\mathbf{x}, \mathbf{x}') = -1/(4\pi|\mathbf{x} - \mathbf{x}'|)$ is \emph{instantaneous}: the potential at time $t$ depends on the noise source at the same time $t$, with no retardation. (Retardation corrections are $\mathcal{O}(v^2/c^2)$, negligible for all non-relativistic systems.) The temporal structure of $C_V(\tau)$ is therefore controlled entirely by the noise kernel:
\begin{equation}
C_V(\tau) = \frac{G^2 m^2}{c^4} \!\iint \!\left[\frac{1}{|\mathbf{x} - \mathbf{x}_L|} - \frac{1}{|\mathbf{x} - \mathbf{x}_R|}\right]\!\! \left[\frac{1}{|\mathbf{x}' - \mathbf{x}_L|} - \frac{1}{|\mathbf{x}' - \mathbf{x}_R|}\right] N_{0000}(\mathbf{x}, t;\, \mathbf{x}', t{+}\tau) \, d^3x \, d^3x'
\end{equation}

For a non-relativistic body in a stationary center-of-mass superposition, the noise kernel $N_{0000}(\mathbf{x}, t;\, \mathbf{x}', t') = \tfrac{1}{2}\langle \{\delta\hat{T}_{00}(\mathbf{x}, t),\, \delta\hat{T}_{00}(\mathbf{x}', t')\} \rangle$ is time-independent, because the mass-density operator $\hat{\rho}(\mathbf{x}, t)$ is constant for a body whose wavepacket spreading time $\sim M\sigma^2/\hbar$ far exceeds all other timescales. (For a microdiamond with $M \sim 10^{-14}$~kg and $\sigma \sim 90$~nm, the spreading time is $\sim 10^6$~s.) Therefore $C_V(\tau) = C_V(0)$ and the double time integral evaluates to $C_V(0)\,t^2$, giving \emph{Gaussian} suppression of coherence:
\begin{equation}
|\rho_{LR}(t)| = |\rho_{LR}(0)| \, \exp\!\left(-\frac{C_V(0)}{2\hbar^2}\,t^2\right)
\label{eq:gaussian_suppression}
\end{equation}
with characteristic coherence timescale
\begin{equation}
\tau_{\text{coh}} = \frac{\hbar\sqrt{2}}{\sqrt{C_V(0)}}
\label{eq:gamma_sc}
\end{equation}

For a point particle with superposition size $d$, dimensional analysis gives $C_V(0) \sim G^2 m^4 / d^2$, and therefore
\begin{equation}
\tau_{\text{coh}} \sim \frac{\hbar d}{G m^2} \sim \tau_{DP}
\label{eq:suppression}
\end{equation}
The self-consistent coherence timescale is the \emph{same order} as the Di\'osi--Penrose timescale. The $G^2$ structure of the variance $C_V(0)$---reflecting the two-step coupling chain (stress-energy fluctuation $\xrightarrow{G}$ metric perturbation $\xrightarrow{G \cdot m}$ potential energy difference)---does not produce a parametric suppression of the timescale, because the Gaussian formula involves $1/\sqrt{C_V} \sim 1/G$ rather than $1/C_V \sim 1/G^2$. Exponential decoherence ($\Gamma \sim C_V \tau_c / \hbar^2$) would give a $G^2$ rate, but this requires a finite noise correlation time $\tau_c$; in the Newtonian limit with stationary matter, no such timescale exists.

The distinction from the Di\'osi--Penrose prediction is qualitative, not parametric: the suppression is \emph{Gaussian} ($e^{-t^2/\tau_{\text{coh}}^2}$) rather than exponential ($e^{-t/\tau_{DP}}$), and the noise is derived from the quantum state rather than postulated. The precise relation between $\tau_{\text{coh}}$ and $\tau_{DP}$ depends on the body geometry and is evaluated in Section~\ref{sec:noise_kernel_eval}.

\subsection{Numerical Consequences}

\begin{center}
\begin{tabular}{lcccc}
\toprule
System & Mass (kg) & $d$ & $\tau_{\text{coh}}$ (Gaussian) & $\tau_{DP} = \hbar / E_\Delta$ \\
\midrule
Electron & $9.1 \times 10^{-31}$ & 1 nm & $\sim 2 \times 10^{27}$ s & $\sim 2 \times 10^{27}$ s \\
Microdiamond & $10^{-14}$ & 100 $\mu$m & $\sim 1.5$ ms$^\dagger$ & $\sim 1.3$ ms$^\dagger$ \\
Planck mass & $2.2 \times 10^{-8}$ & 1 $\mu$m & $\sim 4 \times 10^{-15}$ s & $\sim 3 \times 10^{-15}$ s \\
\bottomrule
\end{tabular}
\end{center}
{\footnotesize $^\dagger$Extended-body $E_\Delta = G(6m^2/(5R) - m^2/d)$ with $R = 90$~nm; see Section~\ref{sec:noise_kernel_eval}.}

\medskip
The self-consistent coherence timescale $\tau_{\text{coh}}$ is the same order as the Di\'osi--Penrose timescale across all mass scales, differing only by an $\mathcal{O}(1)$ prefactor (the ratio is $4\sqrt{2}/5 \approx 1.13$ in the $d \gg R$ limit; see eq.~\ref{eq:tau_coh_exact} below). The two timescales coincide at the Planck mass ($M_P = 2.18 \times 10^{-8}$~kg), where $\kappa \sim 1$ and the framework reaches its validity boundary.

The distinction is qualitative: the self-consistent decoherence is Gaussian ($e^{-t^2/\tau_{\text{coh}}^2}$) rather than the exponential ($e^{-t/\tau_{DP}}$) predicted by the Di\'osi--Penrose models. Gaussian and exponential profiles are experimentally distinguishable in principle: the Gaussian is slower at early times ($t \ll \tau_{\text{coh}}$) and faster at late times ($t \gg \tau_{\text{coh}}$).

\subsection{Noise Kernel Evaluation for Experimental Geometries}
\label{sec:noise_kernel_eval}

The point-particle estimate of Section~\ref{sec:decoherence_derivation} leaves the $\mathcal{O}(1)$ prefactor in $\tau_{\text{coh}}$ undetermined. Experimental targets---microdiamonds with $N \sim 10^{12}$ atoms, optomechanically levitated nanospheres, proposed space-based test masses~\cite{maqro}---are composite objects of radius $R$. We now evaluate the noise kernel for these geometries, computing $C_V(0)$ and the resulting coherence timescale with full numerical prefactors. Throughout this subsection, we label each result as \emph{rigorous}, \emph{sketch}, or \emph{conjecture} following the standards of Section~7.

\subsubsection{Noise kernel for a rigid-body superposition {\normalfont (rigorous)}}

Consider a rigid body of mass $m$ with mass-density distribution $\rho(\mathbf{x})$, placed in a center-of-mass superposition $|\psi\rangle = (|L\rangle + |R\rangle)/\sqrt{2}$ with branches separated by $\mathbf{d} = \mathbf{x}_L - \mathbf{x}_R$. In branch $|L\rangle$, the body has density $\rho_L(\mathbf{x}) = \rho(\mathbf{x} - \mathbf{x}_L)$; in branch $|R\rangle$, $\rho_R(\mathbf{x}) = \rho(\mathbf{x} - \mathbf{x}_R)$. The stress-energy operator satisfies $\hat{T}_{00}(\mathbf{x}) = c^2 \rho(\mathbf{x} - \hat{\mathbf{X}}_{\text{CM}})$.

The expectation value is:
\begin{equation}
\langle \hat{T}_{00}(\mathbf{x}) \rangle = \frac{c^2}{2}\bigl[\rho_L(\mathbf{x}) + \rho_R(\mathbf{x})\bigr]
\end{equation}
Computing the noise kernel $N_{0000}(\mathbf{x}, \mathbf{x}') = \langle \hat{T}_{00}(\mathbf{x}) \hat{T}_{00}(\mathbf{x}') \rangle - \langle \hat{T}_{00}(\mathbf{x}) \rangle \langle \hat{T}_{00}(\mathbf{x}') \rangle$ directly:
\begin{equation}
N_{0000}(\mathbf{x}, \mathbf{x}') = \frac{c^4}{4} \, \Delta\rho(\mathbf{x}) \, \Delta\rho(\mathbf{x}')
\label{eq:noise_kernel_extended}
\end{equation}
where $\Delta\rho(\mathbf{x}) \equiv \rho_L(\mathbf{x}) - \rho_R(\mathbf{x})$ is the mass-density difference between branches. This follows from $\langle \hat{T}_{00} \hat{T}_{00}' \rangle = \frac{c^4}{2}(\rho_L \rho_L' + \rho_R \rho_R')$ (each branch contributes its own product) while $\langle \hat{T}_{00} \rangle \langle \hat{T}_{00}' \rangle = \frac{c^4}{4}(\rho_L + \rho_R)(\rho_L' + \rho_R')$.

\subsubsection{Potential difference and the gravitational self-energy {\normalfont (rigorous)}}

In the Newtonian limit, $\delta\Phi(\mathbf{x}) = -(G/c^2) \int \xi_{00}(\mathbf{z}) / |\mathbf{x} - \mathbf{z}| \, d^3z$. Because the noise kernel~\eqref{eq:noise_kernel_extended} factorizes, the potential correlator also factorizes:
\begin{equation}
\langle \delta\Phi(\mathbf{x}) \, \delta\Phi(\mathbf{y}) \rangle = \frac{G^2}{4} \, U(\mathbf{x}) \, U(\mathbf{y})
\label{eq:phi_correlator}
\end{equation}
where
\begin{equation}
U(\mathbf{x}) \equiv \int \frac{\Delta\rho(\mathbf{z})}{|\mathbf{x} - \mathbf{z}|} \, d^3z
\end{equation}
is the Newtonian potential of the mass-density difference $\Delta\rho$ (without the factor of $-G$). The equal-time potential difference variance is:
\begin{equation}
\langle (\delta V_L - \delta V_R)^2 \rangle = \frac{G^2 m^2}{4} \bigl[U(\mathbf{x}_L) - U(\mathbf{x}_R)\bigr]^2
\label{eq:dV_variance}
\end{equation}

\noindent\textit{Evaluation for a uniform sphere.} For a sphere of radius $R$ and uniform density $\rho_0 = 3m/(4\pi R^3)$, displaced by $d \equiv |\mathbf{d}|$, we need the potential of $\Delta\rho$ evaluated at the branch locations. The self-potential at the center of the body is $U_{\text{self}} = \int \rho_0(u)/|u| \, d^3u = 3m/(2R)$, and the cross-potential for $d > 2R$ (non-overlapping bodies) is $U_{\text{cross}}(d) = m/d$. By symmetry, $U(\mathbf{x}_R) = -U(\mathbf{x}_L)$, giving:
\begin{equation}
U(\mathbf{x}_L) - U(\mathbf{x}_R) = 2\!\left(\frac{3m}{2R} - \frac{m}{d}\right) = m\!\left(\frac{3}{R} - \frac{2}{d}\right)
\label{eq:U_diff}
\end{equation}

This result connects directly to the gravitational self-energy of $\Delta\rho$. Define:
\begin{equation}
E_\Delta \equiv \frac{G}{2} \!\int\!\!\int \frac{\Delta\rho(\mathbf{x}) \, \Delta\rho(\mathbf{x}')}{|\mathbf{x} - \mathbf{x}'|} \, d^3x \, d^3x' = G\!\left(\frac{6m^2}{5R} - \frac{m^2}{d}\right) \quad (d > 2R)
\label{eq:E_delta}
\end{equation}
which is the same integral that determines the Di\'osi--Penrose decoherence timescale $\tau_{DP} = \hbar / E_\Delta$. The potential difference variance~\eqref{eq:dV_variance} can be expressed as:
\begin{equation}
\langle (\delta V_L - \delta V_R)^2 \rangle = \frac{G^2 m^2}{4}\!\left(\frac{3m}{R} - \frac{2m}{d}\right)^{\!2} = \frac{25}{16} \, E_\Delta^2 \!\left(1 + \mathcal{O}(R/d)\right)
\end{equation}

\subsubsection{Gravitational form factor {\normalfont (rigorous)}}

The Fourier transform of the body's density profile defines a gravitational form factor:
\begin{equation}
\tilde{F}(k) \equiv \frac{1}{m} \int \rho(\mathbf{x}) \, e^{-i\mathbf{k}\cdot\mathbf{x}} \, d^3x
\end{equation}
For a uniform sphere of radius $R$:
\begin{equation}
\tilde{F}(kR) = \frac{3[\sin(kR) - kR \cos(kR)]}{(kR)^3}
\label{eq:form_factor}
\end{equation}
with $\tilde{F}(0) = 1$ (normalization) and $\tilde{F}(kR) \to 0$ for $kR \gg 1$. The gravitational self-energy integral~\eqref{eq:E_delta} can be written in Fourier space as:
\begin{equation}
E_\Delta = \frac{2 G m^2}{\pi} \int_0^\infty |\tilde{F}(kR)|^2 \left(1 - \frac{\sin kd}{kd}\right) dk
\label{eq:E_delta_fourier}
\end{equation}
The factor $(1 - \sin(kd)/(kd))$ encodes the incomplete overlap between branches: it vanishes at $k = 0$ (reflecting mass conservation, $\int \Delta\rho \, d^3x = 0$) and saturates to 1 for $kd \gg 1$. The form factor $|\tilde{F}|^2$ provides a UV cutoff at $k \sim 1/R$.

\medskip\noindent\textit{Three regimes:}
\begin{itemize}
\item $d \gg R$: The integral is dominated by $k \sim 1/R$, where $(1 - \sin kd/(kd)) \approx 1$. The form factor controls the result: $E_\Delta \approx 6Gm^2/(5R)$, the gravitational self-energy of the body. This is the point-particle limit for the \emph{body} (not the atoms).

\item $d \sim R$: Both the form factor and the overlap factor contribute. $E_\Delta$ transitions smoothly from the self-energy--dominated regime to the interaction regime.

\item $d \ll R$: The overlap factor restricts the integral to $k \gtrsim 1/d \gg 1/R$. In this regime $|\tilde{F}|^2 \ll 1$, and $E_\Delta \propto d^2/R^5$, parametrically suppressed by $(d/R)^2$ relative to the self-energy.
\end{itemize}

For all current experimental proposals---microdiamonds with $d/R \sim 2{,}800$, levitated nanospheres with $d/R \sim 5$--$10$, and MAQRO test masses with $d/R \sim 0.5$---the relevant regime ranges from $d \gg R$ (BMV) to $d \sim R$ (MAQRO). Only for MAQRO does the form factor provide significant corrections.

\subsubsection{Rigid body vs.\ independent atoms {\normalfont (rigorous)}}

The factorized noise kernel~\eqref{eq:noise_kernel_extended} assumes all $N$ atoms move coherently in the center-of-mass superposition. This gives $E_\Delta \propto m_{\text{total}}^2$. If instead the atoms moved independently (each in its own superposition), the stress-energy fluctuations would be uncorrelated between atoms, and the noise kernel would decompose as a sum over atomic contributions: $N_{0000} = \sum_{i=1}^N N_{0000}^{(i)}$, giving $E_\Delta \propto N m_{\text{atom}}^2 = m_{\text{total}} \cdot m_{\text{atom}}$---a factor of $N$ smaller.

The rigid-body scaling ($m_{\text{total}}^2$) is appropriate for the center-of-mass superpositions created in all proposed experiments~\cite{bose,maqro}: the spin-dependent magnetic gradient (BMV) or coherent scattering (MAQRO) displaces the entire crystal as a unit. The $m_{\text{total}}^4$ scaling of $\Gamma_{\text{sc}}$ (two factors of $m^2$, one from the noise kernel and one from the gravitational coupling) is therefore correct for these geometries.

\subsubsection{Numerical evaluation {\normalfont (rigorous numerics, sketch for $\Gamma_{\text{sc}}$)}}

We evaluate $E_\Delta$ and the resulting timescales for three experimental platforms. For each, we compute both the Di\'osi--Penrose timescale $\tau_{DP} = \hbar/E_\Delta$ using the extended-body self-energy~\eqref{eq:E_delta} and the self-consistent Gaussian coherence timescale $\tau_{\text{coh}} = 4\sqrt{2}\,\hbar / (5\,E_\Delta)$ from eqs.~\eqref{eq:gamma_sc} and~\eqref{eq:tau_coh_exact}.

\medskip
\begin{center}
\begin{tabular}{lccccccc}
\toprule
Platform & $m$ (kg) & $R$ & $d$ & $d/R$ & $\tau_{DP}$ & $\tau_{\text{coh}}$ & Profile \\
\midrule
Microdiamond (BMV) & $10^{-14}$ & 90 nm & 250 $\mu$m & 2{,}800 & 1.2 ms & 1.3 ms & Gaussian \\
Nanosphere (levitated) & $5 \times 10^{-18}$ & 100 nm & 500 nm & 5 & 6{,}300 s & 7{,}100 s & Gaussian \\
MAQRO test mass & $1.7 \times 10^{-17}$ & 200 nm & 100 nm & 0.5 & 10{,}700 s & 12{,}100 s & Gaussian \\
\bottomrule
\end{tabular}
\end{center}
\medskip

\noindent Several features are notable:

\begin{enumerate}
\item The self-consistent coherence timescale $\tau_{\text{coh}}$ and the Di\'osi--Penrose timescale $\tau_{DP}$ are the same order for all platforms, differing by the universal $\mathcal{O}(1)$ prefactor $4\sqrt{2}/5 \approx 1.13$. Both are determined by the gravitational self-energy $E_\Delta$. The distinction is the temporal profile: Gaussian ($e^{-t^2/\tau_{\text{coh}}^2}$) versus exponential ($e^{-t/\tau_{DP}}$).

\item The extended-body $\tau_{DP} \approx 1.2$~ms for the BMV microdiamond (dominated by the gravitational self-energy $6Gm^2/(5R)$) is much shorter than the point-particle estimate $\tau_{DP}^{(\text{pt})} = \hbar d/(Gm^2) \approx 4$~s at $d = 250\;\mu$m. This factor-of-$\sim\!3{,}300$ difference reflects the dominance of the self-energy over the interaction energy when $d \gg R$.

\item For MAQRO ($d/R \approx 0.5$), the form factor~\eqref{eq:form_factor} suppresses $E_\Delta$ by an additional factor of $\sim 12$ relative to the $d \gg R$ limit. The integral~\eqref{eq:E_delta_fourier} must be evaluated numerically for this geometry.
\end{enumerate}

\subsubsection{Coherence timescale for extended bodies {\normalfont (rigorous)}}

The equal-time potential difference variance~\eqref{eq:dV_variance}, combined with the temporal analysis of Section~\ref{sec:decoherence_derivation}, gives the coherence timescale. In the Newtonian limit, the Poisson Green function is instantaneous, so $C_V(\tau)$ inherits its temporal structure from the noise kernel~\eqref{eq:noise_kernel_extended}. For a non-relativistic body in a stationary superposition, the noise kernel is time-independent (Section~\ref{sec:decoherence_derivation}), so $C_V(\tau) = C_V(0)$.

From~\eqref{eq:dV_variance} and the evaluation for a uniform sphere:
\begin{equation}
C_V(0) = \frac{G^2 m^2}{4}\!\left(\frac{3m}{R} - \frac{2m}{d}\right)^{\!2} = \frac{25}{16}\,E_\Delta^2\!\left(1 + \mathcal{O}(R/d)\right)
\end{equation}
The Gaussian coherence timescale~\eqref{eq:gamma_sc} is therefore:
\begin{equation}
\tau_{\text{coh}} = \frac{\hbar\sqrt{2}}{\sqrt{C_V(0)}} = \frac{4\sqrt{2}}{5}\,\frac{\hbar}{E_\Delta}\!\left(1 + \mathcal{O}(R/d)\right) \approx 1.13\,\tau_{DP}
\label{eq:tau_coh_exact}
\end{equation}
The self-consistent coherence timescale exceeds $\tau_{DP}$ by a factor of $4\sqrt{2}/5$, independent of mass, radius, or separation.

This result resolves the temporal structure of the decoherence: it is Gaussian, not exponential, and the timescale is set by the gravitational self-energy $E_\Delta$ of the difference mass distribution---the same quantity that determines the Di\'osi--Penrose timescale. The $G^2$ structure of the variance $C_V(0)$ is real (it reflects the two-step noise propagation chain), but the Gaussian formula $\tau_{\text{coh}} \propto 1/\sqrt{C_V} \propto 1/G$ removes one power of $G$, yielding the same $G$-scaling as $\tau_{DP}$.

\medskip\noindent\textit{Robustness {\normalfont (sketch)}.} The static-noise result ($C_V(\tau) = C_V(0)$) holds as long as two conditions are satisfied: (a)~the wavepacket spreading time $M\sigma^2/\hbar$ exceeds the decoherence timescale, and (b)~internal degrees of freedom (phonons, rotational modes) do not introduce temporal correlations in the noise kernel at the relevant frequency scale $\sim 1/\tau_{\text{coh}}$. Condition~(a) is satisfied for all macroscopic bodies: for the microdiamond, $M\sigma^2/\hbar \sim 10^6$~s $\gg \tau_{\text{coh}} \sim 10^{-3}$~s. Condition~(b) is satisfied when the internal excitation spectrum has no modes at the gravitational frequency $\omega_g \sim E_\Delta/\hbar \sim 10^3$~rad/s; for rigid crystalline solids, the lowest acoustic phonon frequencies ($\sim 10^9$~Hz) are far above $\omega_g$, so internal dynamics decouple. If condition~(b) were violated---for instance, by a soft body with vibrational modes near $\omega_g$---the noise kernel would acquire temporal structure and the decoherence could transition from Gaussian toward exponential. This would require detailed modeling of the body's internal stress-energy fluctuations, a calculation beyond the rigid-body approximation used here.

%======================================================================
\section{Relation to the Di\'osi--Penrose Prediction}
\label{sec:dp}
%======================================================================

The Di\'osi--Penrose decoherence timescale $\tau_{DP} = \hbar / E_\Delta$ is widely cited as a prediction of gravitational decoherence, and several experimental programs are designed to test it at mesoscopic scales~\cite{bose}. It is sometimes assumed that this timescale requires physics beyond the semiclassical Einstein equation---in particular, that one must either postulate fundamental noise in the gravitational potential~\cite{diosi} or invoke the ill-definedness of the time-translation operator for superposed geometries~\cite{penrose}.

The analysis of Sections~\ref{sec:decoherence_derivation} and~\ref{sec:noise_kernel_eval} shows a more nuanced picture. The semiclassical self-consistency equation plus its stochastic extension \emph{does} produce gravitational decoherence at a timescale comparable to $\tau_{DP}$: specifically, Gaussian decoherence with $\tau_{\text{coh}} = (4\sqrt{2}/5)\,\tau_{DP}$. The $G^2$ structure of the noise variance is real---but it manifests as a second power of $G$ in the \emph{variance} $C_V(0) \propto G^2$, not as a suppression of the \emph{timescale} $\tau_{\text{coh}} \propto 1/\sqrt{C_V} \propto 1/G$.

What distinguishes the self-consistent prediction from Di\'osi--Penrose is not the timescale but the temporal profile:
\begin{itemize}
\item \textbf{Di\'osi--Penrose}: Exponential decay, $|\rho_{LR}(t)| \sim e^{-t/\tau_{DP}}$. Arises from postulated white noise ($\delta$-correlated in time) in the gravitational potential, with amplitude set by the gravitational self-energy.
\item \textbf{Self-consistent (this framework)}: Gaussian decay, $|\rho_{LR}(t)| \sim e^{-t^2/\tau_{\text{coh}}^2}$. Arises from static noise (time-independent in the Newtonian limit for stationary matter) derived from quantum stress-energy fluctuations.
\end{itemize}

The origin of the different temporal profiles is clear: Di\'osi's model postulates a stochastic process with a white-noise spectrum, which produces exponential decoherence. The self-consistent calculation derives the noise from the quantum state; for a stationary center-of-mass superposition in the Newtonian limit, this noise is static (Section~\ref{sec:decoherence_derivation}), producing Gaussian decoherence. The physical content of the Di\'osi postulate---over and above what follows from semiclassical self-consistency---is therefore the temporal correlator of the noise, not its spatial structure or amplitude.

This sharpens the theoretical landscape:
\begin{itemize}
\item Experiments testing gravitational decoherence at the $\tau_{DP}$ timescale \emph{are} testing semiclassical gravity, not physics beyond it.
\item The discriminating observable is the \emph{temporal profile} of the decoherence: exponential (Di\'osi--Penrose, requiring postulated noise) versus Gaussian (self-consistent stochastic gravity).
\item Observation of exponential decoherence at rate $E_\Delta/\hbar$ would support the Di\'osi model's additional postulate of white gravitational noise. Observation of Gaussian decoherence at the same timescale would support the self-consistency framework.
\item Absence of any gravitational decoherence at the $E_\Delta$ timescale would falsify both frameworks.
\end{itemize}

%======================================================================
\section{Experimental Predictions}
\label{sec:experiment}
%======================================================================

\subsection{BMV Entanglement}

The Bose--Marletto--Vedral experiment~\cite{bose,marletto_vedral} places two masses in spatial superposition and tests for gravitational entanglement.

\begin{theorem}[Separability preservation---first-quantized matter]
In the linearized regime with first-quantized matter, semiclassical gravitational evolution preserves the separability of quantum states.
\end{theorem}

\noindent\emph{Proof.} For a product state $\rho = \rho_1 \otimes \rho_2$, the stress-energy decomposes as $\langle \hat{T}_{\mu\nu} \rangle = \langle \hat{T}_{\mu\nu} \rangle_1 + \langle \hat{T}_{\mu\nu} \rangle_2$. In the linearized Einstein equation, $h_{\mu\nu} = h^{(1)}_{\mu\nu} + h^{(2)}_{\mu\nu}$, and the Hamiltonian factorizes into single-particle operators plus a c-number interaction potential from the smeared distributions. A c-number potential cannot generate entanglement (LOCC theorem). The nonlinear correction is of order $Gm/c^2 D \sim 10^{-38}$ for BMV parameters. \hfill$\square$

When matter is described within full QFT, Aziz and Howl~\cite{aziz_howl} have shown that virtual matter propagators can transmit quantum information through a classical gravitational interaction, generating entanglement at fourth order in perturbation theory. The effect scales as $\vartheta \propto G^2 m^2 M^3 R \, t / \hbar^3 d$, suppressed relative to the quantum gravity prediction ($\varphi \propto GM^2 t / \hbar d$) by additional powers of mass.

\textbf{Prediction:} The BMV experiment will observe either (a)~no entanglement, or (b)~entanglement at the suppressed classical-gravity rate from virtual matter propagators. It will not observe entanglement at the rate predicted by quantum gravity. Observation of entanglement at the quantum gravity rate would falsify this framework.

This prediction is independent of the decoherence timescale and does not require the Di\'osi--Penrose rate.

\subsection{Gravitational Decoherence Experiments}

The framework predicts that experiments searching for gravitational decoherence at the Di\'osi--Penrose timescale~\cite{diosi,penrose} will observe Gaussian decoherence with coherence timescale $\tau_{\text{coh}} \approx 1.13\,\tau_{DP}$ (eq.~\ref{eq:tau_coh_exact}). For a microdiamond ($m \sim 10^{-14}$~kg, $R \sim 90$~nm), the extended-body timescale is $\tau_{\text{coh}} \approx 1.3$~ms.

The critical test involves placing a mass $m \sim 10^{-14}$~kg in superposition with all environmental decoherence suppressed, then measuring the fringe visibility as a function of time:
\begin{itemize}
\item If fringe visibility decays as $e^{-t/\tau}$ (exponential) with $\tau \sim \hbar/E_\Delta$, the Di\'osi model's postulate of white gravitational noise is supported.
\item If fringe visibility decays as $e^{-t^2/\tau^2}$ (Gaussian) with $\tau \sim \hbar/E_\Delta$, the self-consistency framework is supported.
\item If no anomalous decoherence is observed beyond environmental sources at the $E_\Delta$ timescale, both frameworks are falsified.
\end{itemize}

\subsection{Distinguishing from Alternative Models}

\begin{center}
\begin{tabular}{lcccc}
\toprule
Model & Timescale & $G$-scaling & Profile & Noise origin \\
\midrule
This framework & $\hbar / E_\Delta$ & $1/G$ & Gaussian & Derived (quantum $\hat{T}_{\mu\nu}$) \\
Di\'osi--Penrose & $\hbar / E_\Delta$ & $1/G$ & Exponential & Postulated (white) \\
Kafri--Taylor--Milburn & $\sim \hbar D^5/(G^2 m^3)$ & $1/G^2$ & Exponential & Classical channel \\
\bottomrule
\end{tabular}
\end{center}

The key discriminator between this framework and Di\'osi--Penrose is the temporal profile (Gaussian vs.\ exponential), not the timescale or $G$-scaling, which are shared. Both predict decoherence at the $E_\Delta$ timescale for extended bodies. The Kafri--Taylor--Milburn model~\cite{kafri} remains parametrically distinct, predicting a $G^2$ timescale. In all three models, the noise variance scales as $G^2$; the different timescale scalings arise from different assumptions about the noise correlation time.

%======================================================================
\section{Relation to Existing Work}
%======================================================================

\textbf{Semiclassical gravity} (M\o ller, Rosenfeld): Our starting point. We treat the SCE and its stochastic extension as the sufficient description of gravity below the Planck scale, rather than an approximation to a deeper theory.

\textbf{Stochastic gravity} (Hu, Verdaguer~\cite{hu_verdaguer,phillips_hu}): The Einstein--Langevin equation is the natural fluctuation extension of our framework. Our decoherence calculation (Section~\ref{sec:decoherence}) is performed within this formalism.

\textbf{Penrose collapse}~\cite{penrose}: Predicts the same decoherence timescale as Di\'osi but from a different argument. Our framework recovers a comparable timescale but with Gaussian rather than exponential temporal profile (Section~\ref{sec:dp}).

\textbf{Di\'osi model}~\cite{diosi}: Postulates fundamental white noise at the Newtonian potential level. Our framework derives the noise from quantum stress-energy fluctuations, obtaining the same decoherence timescale but with different temporal structure (Gaussian vs.\ exponential). The physical content of Di\'osi's additional postulate is the temporal correlator (white noise), not the spatial structure or amplitude.

\textbf{Postquantum classical gravity} (Oppenheim~\cite{oppenheim}): The most developed modern framework for classical-gravity-coupled-to-quantum-matter. Oppenheim derives a decoherence-diffusion trade-off that constrains any theory in which gravity couples to quantum matter through a classical channel. Our framework is not in this class: the noise kernel (eq.~\ref{eq:noise_kernel}) is a quantum object---the symmetrized two-point correlator of stress-energy fluctuations---and the stochastic metric fluctuations it sources are isomorphic to tree-level graviton exchange, not to a classical communication channel. The framework therefore escapes Oppenheim's trade-off rather than satisfying it. What the fixed-point analysis adds is a constructive existence proof with a quantitative contraction constant $\kappa \sim (m/M_P)^2$, giving a derived Planck-scale validity boundary rather than an imposed one. The null BMV prediction is shared between frameworks.

\textbf{Inflationary spectrum from stochastic gravity} (Roura, Verdaguer~\cite{roura_verdaguer}): Demonstrates that the stochastic gravity formalism reproduces the full spectrum of cosmological perturbations---scalar and tensor---without quantizing the metric. This removes a standard objection that semiclassical gravity cannot account for the inflationary perturbation spectrum, and supports the interpretation of the framework as tree-level quantum gravity.

\textbf{Classical gravity entanglement} (Aziz, Howl~\cite{aziz_howl}): Shows that QFT matter on a classical gravitational background can generate entanglement through virtual matter propagators. Compatible with our ontology.

\textbf{Page--Geilker experiment}~\cite{page_geilker}: Found that the gravitational field tracks definite measurement outcomes rather than expectation values. In our block-universe ontology (Axiom~3), this is expected: the global self-consistent solution includes the definite outcome.

%======================================================================
\section{Open Problems}
\label{sec:open}
%======================================================================

\begin{enumerate}
\item \textbf{Effective initial value formulation.} Axiom~3 states that the solution is global. Practical physics needs a local formulation---solving on patches and stitching together.

\item \textbf{Rigorous Schauder application.} The non-perturbative existence argument requires specifying the convex compact $\mathcal{F}$-invariant subset, including gauge fixing in Lorentzian signature.

\item \textbf{Noise kernel for interacting matter.} The noise kernel evaluation of Section~\ref{sec:noise_kernel_eval} treats the body as a rigid, non-interacting object. Internal stress-energy fluctuations from strong interactions (QCD composites, phonon modes near the gravitational frequency $\omega_g \sim E_\Delta/\hbar$) could introduce temporal structure in the noise kernel, modifying the Gaussian profile and affecting the numerical prefactor. This is relevant for soft or composite bodies but not for rigid crystalline solids (see Section~\ref{sec:noise_kernel_eval}).

\item \textbf{Macroscopic definiteness.} The framework does not currently derive the emergence of definite macroscopic outcomes from the self-consistency condition alone. The stochastic extension provides Gaussian decoherence at the $E_\Delta$ timescale, which suppresses coherence for mesoscopic masses but does not by itself select a definite outcome. Whether the global constraint (Axiom~3) provides additional selection of definite outcomes remains open.

\item \textbf{Black hole evaporation and the Page time.} The framework inherits the Hawking calculation, which predicts monotonically increasing radiation entropy. For large black holes---where curvature is small and $\kappa \ll 1$, well within the framework's stated validity domain---this conflicts with unitarity at the Page time~\cite{page_curve}. The island formula~\cite{penington,aemm} resolves this via topology-changing saddles in the gravitational path integral, but topology change amounts to summing over geometries, which the framework does not include.

This reveals that the framework's validity requires two conditions, not one: (a)~$\kappa \ll 1$ (quantum backreaction is perturbatively small), and (b)~the observable in question must be computable on a single spacetime topology. The Page curve violates condition~(b) despite satisfying condition~(a). We regard this as the sharpest limitation of the tree-level description. Condition~(b) is not self-diagnosable: determining whether topology change contributes to a given observable requires the full theory of quantum gravity, which the framework does not provide.

\item \textbf{Lorentzian signature and conformal symmetry breaking.} The physical content of the timelike direction---rest frames, proper time, timelike geodesics---depends entirely on the existence of massive fields. In a purely conformal universe with no symmetry breaking, the $(-,+,+,+)$ signature would have no observable consequences: no field would have a rest frame and no proper time could be defined for any physical degree of freedom. This observation appears in Penrose's conformal cyclic cosmology~\cite{penrose_ccc}, where the far-future decay of all massive particles renders the timelike/null distinction physically irrelevant. Wheeler~\cite{wheeler_biconformal} has shown, conversely, that gauging the conformal group $SO(4,2)$ on Euclidean space dynamically produces Lorentzian signature on submanifolds. Whether the self-consistency fixed point of Section~\ref{sec:fixedpoint} \emph{requires} Lorentzian signature when the field content includes massive fields---i.e., whether the Banach contraction fails on a Riemannian manifold---is an open question that connects the present framework to these programs.
\end{enumerate}

%======================================================================
\section{Summary}
%======================================================================

We have developed the consequences of a single thesis: physical law is self-consistency, and gravity is the constraint that enforces it. The semiclassical Einstein equation and its stochastic extension implement this constraint at tree level. The block spacetime---the complete four-dimensional solution---is the ontology.

The mathematical content of this thesis is:
\begin{enumerate}
\item \textbf{Self-consistent solutions exist} with quantitative control---exactly in cosmology, perturbatively near all classical solutions with contraction constant $\kappa \sim (m/M_P)^2 \ll 1$, and conditionally in the general case. The Planck scale emerges as the natural validity boundary where $\kappa \to 1$.

\item \textbf{The Di\'osi--Penrose timescale follows from self-consistency, but with Gaussian temporal profile.} The stochastic extension yields Gaussian decoherence at timescale $\tau_{\text{coh}} \approx 1.13\,\hbar/E_\Delta$---the same order as the Di\'osi--Penrose prediction. The $G^2$ scaling of the noise variance is real, but the Gaussian formula converts this to a $1/G$ timescale. The distinguishing prediction is the temporal profile: Gaussian (from static noise derived from the quantum state) rather than exponential (from white noise postulated in the Di\'osi model).

\item \textbf{BMV entanglement experiments will not observe entanglement at the quantum gravity rate.} Semiclassical evolution preserves separability at the first-quantized level; QFT corrections allow only a suppressed channel distinguishable from quantum gravity predictions.
\end{enumerate}

The framework is falsifiable: observation of BMV entanglement at the quantum gravity rate would refute it, as would the complete absence of gravitational decoherence at the $E_\Delta$ timescale. Observation of exponential (rather than Gaussian) decoherence at this timescale would support the Di\'osi model's additional postulate over the self-consistency framework. Its sharpest known limitation is black hole evaporation past the Page time, where topology change---absent from the tree-level description---becomes essential (Section~\ref{sec:open}).

%======================================================================
\begin{thebibliography}{99}

\bibitem{bose} S.~Bose \textit{et al.}, Spin entanglement witness for quantum gravity, \textit{Phys.\ Rev.\ Lett.}\ \textbf{119}, 240401 (2017).

\bibitem{marletto_vedral} C.~Marletto and V.~Vedral, Gravitationally induced entanglement between two massive particles is sufficient evidence of quantum effects in gravity, \textit{Phys.\ Rev.\ Lett.}\ \textbf{119}, 240402 (2017).

\bibitem{starobinsky} A.~A.~Starobinsky, A new type of isotropic cosmological models without singularity, \textit{Phys.\ Lett.\ B}\ \textbf{91}, 99 (1980).

\bibitem{capper_duff} D.~M.~Capper and M.~J.~Duff, Trace anomalies in dimensional regularization, \textit{Nuovo Cimento A}\ \textbf{23}, 173 (1974).

\bibitem{horowitz} G.~T.~Horowitz, Semiclassical relativity: The weak-field limit, \textit{Phys.\ Rev.\ D}\ \textbf{21}, 1445 (1980).

\bibitem{phillips_hu} N.~G.~Phillips and B.~L.~Hu, Noise kernel in stochastic gravity and stress energy bitensor, \textit{Phys.\ Rev.\ D}\ \textbf{63}, 104001 (2001).

\bibitem{hu_verdaguer} B.~L.~Hu and E.~Verdaguer, Stochastic gravity: Theory and applications, \textit{Living Rev.\ Relativ.}\ \textbf{11}, 3 (2008).

\bibitem{diosi} L.~Di\'osi, A universal master equation for the gravitational violation of quantum mechanics, \textit{Phys.\ Lett.\ A}\ \textbf{120}, 377 (1987).

\bibitem{penrose} R.~Penrose, On gravity's role in quantum state reduction, \textit{Gen.\ Rel.\ Grav.}\ \textbf{28}, 581 (1996).

\bibitem{radzikowski} M.~J.~Radzikowski, Micro-local approach to the Hadamard condition, \textit{Commun.\ Math.\ Phys.}\ \textbf{179}, 529 (1996).

\bibitem{choquet_bruhat} Y.~Choquet-Bruhat, Th\'eor\`eme d'existence pour certains syst\`emes d'\'equations aux d\'eriv\'ees partielles non lin\'eaires, \textit{Acta Math.}\ \textbf{88}, 141 (1952).

\bibitem{schauder} J.~Schauder, Der Fixpunktsatz in Funktionalr\"aumen, \textit{Studia Math.}\ \textbf{2}, 171 (1930).

\bibitem{kafri} D.~Kafri, J.~M.~Taylor, and G.~J.~Milburn, A classical channel model for gravitational decoherence, \textit{New J.\ Phys.}\ \textbf{16}, 065020 (2014).

\bibitem{aziz_howl} J.~Aziz and R.~Howl, Classical theories of gravity produce entanglement, \textit{Nature}\ \textbf{646}, 813--817 (2025).

\bibitem{page_geilker} D.~N.~Page and C.~D.~Geilker, Indirect evidence for quantum gravity, \textit{Phys.\ Rev.\ Lett.}\ \textbf{47}, 979 (1981).

\bibitem{oppenheim} J.~Oppenheim, A postquantum theory of classical gravity?, \textit{Phys.\ Rev.\ X}\ \textbf{13}, 041040 (2023).

\bibitem{roura_verdaguer} A.~Roura and E.~Verdaguer, Cosmological perturbations from stochastic gravity, \textit{Phys.\ Rev.\ D}\ \textbf{78}, 064010 (2008).

\bibitem{page_curve} D.~N.~Page, Average entropy of a subsystem, \textit{Phys.\ Rev.\ Lett.}\ \textbf{71}, 1291 (1993).

\bibitem{penington} G.~Penington, Entanglement wedge reconstruction and the information problem, \textit{J.\ High Energy Phys.}\ \textbf{2020}, 002 (2020).

\bibitem{aemm} A.~Almheiri, N.~Engelhardt, D.~Marolf, and H.~Maxfield, The entropy of bulk quantum fields and the entanglement wedge of an evaporating black hole, \textit{J.\ High Energy Phys.}\ \textbf{2019}, 063 (2019).

\bibitem{penrose_ccc} R.~Penrose, \textit{Cycles of Time: An Extraordinary New View of the Universe} (Bodley Head, London, 2010).

\bibitem{wheeler_biconformal} S.~T.~Spencer and J.~T.~Wheeler, The existence of time, \textit{Int.\ J.\ Geom.\ Methods Mod.\ Phys.}\ \textbf{8}, 273 (2011); arXiv:1512.01729.

\bibitem{maqro} R.~Kaltenbaek \textit{et al.}, Macroscopic quantum resonators (MAQRO), \textit{Exp.\ Astron.}\ \textbf{34}, 123 (2012).

\end{thebibliography}

\end{document}

